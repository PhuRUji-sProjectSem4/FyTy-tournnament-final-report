\chapter{\ifenglish Conclusions and Discussions\else บทสรุปและข้อเสนอแนะ\fi}

\section{\ifenglish Conclusions\else สรุปผล\fi}

โครงงานนี้เป็นโครงงานเว็ปแอปพลิเคชั่นที่ใช้ในการจัดการทัวร์นาเมนต์ และเก็บบันทึกสถิติเป็นประวัติผลงาน
ทั้งสำหรับผู้จัดการแข่งขันและผู้เล่นเพื่อนำไปพัฒนาต่อยอดในสายอาชีพนี้ได้ดียิ่งขึ้น
โดยตัวโครงงานนี้สามารถตอบโจทย์การใช้งานได้ในระบบนึงจากการไปให้ผู้ทดสอบที่เป็นคนในสาย E-sport 
ในการทดลองใช้จะติดที่การแสดงผลของตารางแข่ง และนำคนลงสายแข่งที่ยังทำให้บางคนสับสนจากตัวเลือก และเนื้อหาที่แสดงมีจำนวนมากและเฉพาะทาง
\section{\ifenglish Challenges\else ปัญหาที่พบและแนวทางการแก้ไข\fi}

ในการทำโครงงานนี้ พบว่าเกิดปัญหาหลักๆ ดังนี้

\begin{enumerate}
    \item ออกแบบ และวางแผนไม่รัดกุมทั้ง UX/UI และดาต้าเบสทำให้ต้องมีการแก้ไขดาต้าเบส และ UX/UI เฉพาะหน้าเพื่อให้ระบบสามารถทำงานได้อย่างถูกต้อง
    \item เวลาในการพัฒนาโครงงานไม่เพียงพอเนื่องจากมีการเปลี่ยนหัวข้อ และจำนวนคนที่น้อย แนวทางการแก้ไขปัญหา ลดปริมาณของงาน และรายละเอียดที่ไม่จำเป็นลงเพื่อให้ใช้เวลาในการพัฒนาที่น้อยลง 
    เพื่อที่จะได้มีเวลานำเว็ปแอปพลิเคชั่นนี้ไปทำการทดสอบ และเก็บข้อมูล เพื่อที่จะได้ข้อมูลข้อเสนอแนะมาแก้ไข และปรับปรุงให้ดียิ่งขึ้นเหมาะกับกันใช้งานจริงมากยิ่งขึ้น
\end{enumerate}

\section{\ifenglish%
Suggestions and further improvements
\else%
ข้อเสนอแนะและแนวทางการพัฒนาต่อ
\fi
}

ข้อเสนอแนะเพื่อพัฒนาโครงงานนี้ต่อไป มีดังนี้

\begin{enumerate}
    \item ทำตารางแข่งแบบแบบรูปภาพเพื่อให้ดูได้ง่ายและเข้าใจได้ง่าย
    \item ทำระบบแจ้งเตือนเมื่อมีคนขอเข้าทีม หรือใกล้ถึงเวลาแข่ง เพื่อจะได้รับรู้และไม่ลืม โดยหากจะทำระบบนี้ให้รองรับกับการใช้งานจะต้องทำในรูปแบบ Application มือถือเพื่อให้การแจ้งเตือนนั้นมีประสิทธิภาพ
    \item ทำให้หน้าเว็ปไซต์ให้ responsive กับทุกอุปกรณ์ เพื่อการใช้งานได้ในทุกที่ทุกเวลา
\end{enumerate}
